\documentclass[twocolumn,english,pra,aps,superscriptaddress,floatfix]{revtex4-1}

\usepackage{amsthm}
\usepackage{amsmath}
\usepackage{graphicx}
\usepackage{amssymb}
%\usepackage{esint}
\usepackage{bm}
\usepackage{latexsym}

\makeatletter
%%%%%%%%%%%%%%%%%%%%%%%%%%%%%% Textclass specific LaTeX commands.
\@ifundefined{textcolor}{}
{%
 \definecolor{BLACK}{gray}{0}
 \definecolor{WHITE}{gray}{1}
 \definecolor{RED}{rgb}{1,0,0}
 \definecolor{GREEN}{rgb}{0,1,0}
 \definecolor{BLUE}{rgb}{0,0,1}
 \definecolor{CYAN}{cmyk}{1,0,0,0}
 \definecolor{MAGENTA}{cmyk}{0,1,0,0}
 \definecolor{YELLOW}{cmyk}{0,0,1,0}
 }

%%%%%%%%%%%%%%%%%%%%%%%%%%%%%% User specified LaTeX commands.
%\numberwithin{equation}{section}

\makeatother

\usepackage{babel}
\usepackage{braket}
\begin{document}



\author{N. Miladinovic}
\affiliation{Department of Physics and Astronomy, McMaster University, 1280 Main
St.\ W., Hamilton, ON, L8S 4M1, Canada} 
\author{D.\ H.\ J.\ O'Dell}
\affiliation{Department of Physics and Astronomy, McMaster University, 1280 Main
St.\ W., Hamilton, ON, L8S 4M1, Canada}

\title{Abraham-Minkowski and the HMW phase}

\begin{abstract}
\label{sec:abstract}
Hey Duncan, just a small sample calculation I'd like you to look over.  Thanks!
\end{abstract}

\pacs{42.50.Ex, 42.50.Pq, 42.60.Da, 42.82.Et, 03.67.Lx}

\maketitle

\section{The HMW term}
\label{sec:hmw}

From Cohen-Tannoudji, we find that 

\begin{equation}
\braket{\mathbf{d}}=2\mathbf{d}_{\mathrm{ab}}U\cos\left(\omega_L t\right)
\end{equation}
where
\begin{equation}
U=\frac{\Omega_1}{2}\frac{\delta_L}{\delta_L^2+\frac{\Gamma^2}{4}+\frac{\Omega_1^2}{2}}
\end{equation}
Now we want to maximize U, so we set $\delta=\Omega_1$ which gives
\begin{equation}
\braket{\mathbf{d}}=\frac{2}{3}\mathbf{d}_{\mathrm{ab}}\cos\left(\omega_L t\right)
\end{equation}
Now for Rubidium, the D2 transition matrix element $\mathbf{d}_{\mathrm{ab}}$ is approximately $3.6\times 10^{-29}$ $\mathrm{C\cdot m}$.  We want to know the size of $\braket{\mathbf{d}}\cdot \mathbf{E}$, where $\mathbf{E}=E_0\cos\left(\omega_L t\right)$.
  We first get rid of the $\cos^2$ term by integrating over time, which will yield a factor of 1/2.  
We now look at the value of the intensity and electric field needed. A power of 10 mW, focused to a spot size of diameter $10$ $\mathrm{\mu m}$  yields an intensity  of $I=10^8$ $\mathrm{W/m^2}$.  We find the corresponding electric field amplitude is $E_0=2.7\times10^5$ V/m.
Then we find that
\begin{equation}
\frac{1}{2}\braket{\mathbf{d}}\cdot \mathbf{E}=4.9\times 10^{-24} \mathrm{J}
\end{equation}
We now can find the size of the HMW term over a 1 second period (t=1 s)
\begin{equation}
\frac{2\mathbf{d}\cdot \mathbf{E}n_r kt}{m}=1.8
\end{equation}
Here we have used the mass of Rubidium ($1.4\times 10^{-25}$ Kg).  This number is large enough for us to see.

The corresponding Rabi frequency for these values are
\begin{equation}
\Omega_1=\frac{\mathbf{d}_{\mathrm{ab}}E_0}{\hbar}=97 GHz
\end{equation}
This is what suggested to me that we could not use $d=\alpha E$ as we are no longer in the weak field limit and can no longer treat this linearly. This is all done for a 2-level atom.  Right now I'm working on the 3 level case to make sure we can get zero absorption.

\end{document}
