\documentclass[aps,pra,showpacs,twocolumn]{revtex4}
\usepackage{graphicx}
\usepackage{epstopdf} 
\usepackage{amsmath}
\usepackage{amsfonts}



\begin{document}

\date{\today}
\author{N. Miladinovic}
\affiliation{Department of Physics and Astronomy, McMaster University, 1280 Main St.\ W., Hamilton, ON, L8S 4M1, Canada}
\author{D. H. J. O'Dell}
\affiliation{Department of Physics and Astronomy, McMaster University, 1280 Main St.\ W., Hamilton, ON, L8S 4M1, Canada}

\title{Wave function in an elliptical ring trap}

\begin{abstract}
We study an ellipse in polar coordinates, calculate its arc length, and then describe the wave function for a particle living on an ellipse. 
\end{abstract}

%\pacs{03.75.Lm, 03.65.Ta, 67.85.Pq, 05.30.Rt}

\maketitle

\section{Introduction}
The equation for an ellipse in cartesian coordinates is
\begin{equation}
\frac{x^2}{a^2}+\frac{y^2}{b^2}=1
\end{equation}
where $a$ and $b$ are the semimajor and semiminor axes, respectively, so that $a \ge b$. The eccentricity is defined as 
\begin{equation}
\epsilon \equiv \sqrt{1-b^2/a^2} 
\end{equation}
and lies in the range $0 \le \epsilon <1$.
An ellipse can be described as a parametric curve by letting
\begin{eqnarray}
x & = & a \cos u \\
y & = & b \sin u
\end{eqnarray}
where $0 \le u \le 2 \pi $. Note that $u$, the distance around the ellipse, is \textbf{not} the polar angle $\theta$, which is instead related to the cartesian coordinates by
\begin{eqnarray}
x & = & r \cos \theta \\
y & = & r \sin \theta . 
\end{eqnarray}
The relation between $u$ and $\theta$ is given by
\begin{equation}
 \theta = \arctan \frac{y}{x} = \arctan \left( \frac{b}{a} \tan u \right) .
\end{equation}
The differential arc length is given by
\begin{equation}
ds = \sqrt{\left( \frac{dx}{du}\right)^2+\left( \frac{dy}{du}\right)^2} du
\end{equation}
The arc length (measured from the x-axis) is therefore 
\begin{eqnarray}
s(u) & = & \int_{0}^{u} \sqrt{a^2 \sin^{2} u' + b^2 \cos^{2} u' }  \ du' \\
& = & b \int_{0}^{u} \sqrt{1- \left( 1- \frac{a^2}{b^2} \right) \sin^{2} u'  }  \ du' \\
& = & b \  \mathrm{E} \bigg( u  \bigg \vert 1- \frac{a^2}{b^2} \bigg) = a \  \mathrm{E} \bigg( u  \bigg \vert 1- \frac{b^2}{a^2} \bigg)
\end{eqnarray}
where $\mathrm{E} (\phi \vert  m)$ is the elliptic integral of the 2nd kind (available in \textit{Mathematica}). The length of the perimeter of the ellipse is given by putting $u=2 \pi$. This can also be written as four times one quadrant
\begin{equation}
L= 4 b \  \mathrm{E} \bigg( \pi/2  \bigg \vert 1- \frac{a^2}{b^2} \bigg) = 4 b \  \mathrm{E} \bigg(  1- \frac{a^2}{b^2} \bigg)
\end{equation}
where $E(m)$ is the complete elliptic integral of the second kind.

The radius of the ellipse as a function of $\theta$ is given by 
\begin{equation}
r^2=\frac{a^2 b^2}{b^2 \cos^{2} \theta + a^2 \sin^{2} \theta} \ .
\end{equation}
Just as useful is the radius as a function of $u$, which is given by
\begin{equation}
r^2=x^2+y^2=a^2 \cos^2 u + b^2 \sin^2 u
\end{equation}
or in other words
\begin{equation}
r=a\sqrt{1-(1-b^2/a^2) \sin^2 u}=a \sqrt{1-\epsilon^2 \sin^2 u} \ .
\end{equation}

A single-valued wave function on the ellipse can be written as a Fourier series
\begin{equation}
\psi(u)= \sum_{n} A_{n} \frac{e^{i k_{n} s(u)}}{\sqrt{L}} 
\label{eq:wf}
\end{equation}
where $k_{n}= 2 \pi n / L$.  Note the normalization here is
\begin{equation}
\int_0^L\frac{e^{i (k_{n}-k_{m}) s(u)}}{L}\:ds(u)=\begin{cases}
1 & n=m\\
0 & n\ne m \ . \end{cases}
\end{equation}
 When inserted in the Schr\"{o}dinger equation, we need to take the derivative with respect to the arc length $s$
\begin{equation}
\left(-\frac{\hbar^2}{2m} \frac{d^2}{d s^2} +V(u)\right) \psi = E \psi \ .
\end{equation}
This is due to the fact that in elliptic coordinates the scale factor, which measures the length of the basis vector, is given by
\begin{equation}
h_{\mu}=h_{\nu}=\sqrt{a^2 \mathrm{\sin^{2}} u + b^2 \mathrm{\cos^{2}} u}
\end{equation}
where $u$ and $\nu$ are the elliptic coordinates. We can write the Laplacian in elliptic coordinates as
\begin{equation}
\frac{1}{a^2 \mathrm{\sin^{2}} u + b^2 \mathrm{\cos^{2}} u}\:\frac{d^2}{d u^2}
\end{equation}
Now we can see why the derivative with respect to the arc length is correct as
\begin{eqnarray}
\frac{d^2}{d s^2}=\frac{d u^2}{d s^2}\frac{d^2 }{d u^2}&=&\frac{1}{a^2 \mathrm{\sin^{2}} u + b^2 \mathrm{\cos^{2}} u}\:\frac{d^2 }{d u^2} \nonumber \\
&=&\frac{1}{a^2 \left(1- \epsilon^2\mathrm{\cos^{2}} u\right)}\:\frac{d^2 }{d u^2}
\end{eqnarray}
Let us now look at the Hamiltonian for a particle living on an ellipse in an electromagnetic field of the form
\begin{equation}
\mathrm{E}(r)= \hbar E_0 r e^{\frac{-2r^2}{w_0^2}} 
\end{equation}
where $E_0$ is proportional to the intensity of the laser field, and $w_0$ is the beam waist. The Schr\"{o}dinger equation is
\begin{equation}
i\hbar\frac{\partial}{\partial t}\psi=-\frac{\hbar^2}{2m} \left(\frac{d}{d s}+i E_0 r e^{\frac{-2r^2}{w_0^2}} \right)^2 \psi 
\end{equation}
Let's expand out the Hamiltonian 
\begin{eqnarray}
&&\left(\frac{d}{d s}-i E_0 r e^{\frac{-2r^2}{w_0^2}} \right)^2\psi=  \\ 
&&\frac{d^2}{d s^2}\psi +iE_0e^{\frac{-2r^2}{w_0^2}}\frac{\epsilon^2 \mathrm{\sin{2u}}}{2\sqrt{1-\epsilon^2 \cos^2 u}\sqrt{1-\epsilon^2 \sin^2 u}}\psi \nonumber \\
&&+iE_0e^{\frac{-2r^2}{w_0^2}}\frac{2a^2\epsilon^2 \sin{2u}\sqrt{1-\epsilon^2 \sin^2 u}}{\sqrt{1-\epsilon^2 \cos^2 u}}\psi \nonumber \\
&& -i2E_0\frac{-2r^2}{w_0^2}r\frac{d\psi}{ds}-E_0^2\frac{-4r^2}{w_0^2}r^2\psi 
\end{eqnarray}
We now plug in $\psi(u)=\sum_{n} A_{n} \frac{e^{i k_{n} s(u)}}{\sqrt{L}}$ and integrate out the $A_m$ term by multiply the Schr\"{o}dinger equation by $A_{m} \frac{e^{-i k_{n} s(u)}}{\sqrt{L}}$ and integrating with respect to $s(u)$.  This gives 
\begin{eqnarray}
&&i\left(\frac{2m}{\hbar}\right)\dot{A_m}=+k^2\frac{A_m}{L} \nonumber \\
&& -\sum_m A_m\frac{i E_0}{L} \int_0^{2\pi} e^{\frac{-2r^2}{w_0^2}}\frac{a\epsilon^2 \sin{2u}}{2\sqrt{1-\epsilon^2 \sin^2 u}}e^{ik(n-m)s(u)/L}\:du \nonumber \\
&&-\sum_m A_m \frac{i E_0}{L}\int_0^{2\pi} e^{\frac{-2r^2}{w_0^2}}\left(\frac{2ra^3\epsilon^2 \sin{2u}}{w_0^2}\right)e^{ik(n-m)s(u)/L}\:du \nonumber \\
&&-\sum_m A_m \frac{2kE_0}{L}\int_0^{2\pi} e^{\frac{-2r^2}{w_0^2}}re^{ik(n-m)s(u)/L}\:du \nonumber \\
&&+\sum_m A_m\frac{E_0^2}{L}\int_0^{2\pi}e^{\frac{-4r^2}{w_0^2}}ar^2\sqrt{1-\epsilon^2 \cos^2 u}e^{ik(n-m)s(u)/L}\:du \nonumber \\
\end{eqnarray}
where $r=a \sqrt{1-\epsilon^2 \mathrm{\sin^2} u} $ and $k=2m\pi/L$.  Note we have made use of the change of variable $ds \rightarrow  a\sqrt{1-\epsilon^2 \cos^2 u}du$ in the integral which also changed the limits of integration from  $(0,L)$ to $(0,2\pi)$.
Now we wish to write the Hamiltonian in terms of the basis states $\frac{e^{-i k_{n} s(u)}}{\sqrt{L}}$.  This is given by
\begin{eqnarray}
&&\left(\frac{2m}{\hbar^2}\right)H_{n,m}=k^2\delta_{n,m} \nonumber \\
&& -\frac{i E_0}{L} \int_0^{2\pi} e^{\frac{-2r^2}{w_0^2}}\frac{a\epsilon^2 \sin{2u}}{2\sqrt{1-\epsilon^2 \sin^2 u}}e^{ik(n-m)s(u)/L}\:du \nonumber \\
&&-\frac{i E_0}{L}\int_0^{2\pi} e^{\frac{-2r^2}{w_0^2}}\left(\frac{2ra^3\epsilon^2 \sin{2u}}{w_0^2}\right)e^{ik(n-m)s(u)/L}\:du \nonumber \\
&&-\frac{2kE_0}{L}\int_0^{2\pi} e^{\frac{-2r^2}{w_0^2}}re^{ik(n-m)s(u)/L}\:du \nonumber \\
&&+\frac{E_0^2}{L}\int_0^{2\pi}e^{\frac{-4r^2}{w_0^2}}ar^2\sqrt{1-\epsilon^2 \cos^2 u}e^{ik(n-m)s(u)/L}\:du \nonumber \\
\end{eqnarray}
\end{document}